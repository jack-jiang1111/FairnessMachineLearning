\documentclass[11pt]{article}
\usepackage[utf8]{inputenc}
\usepackage{amsmath}
\usepackage{amsfonts}
\usepackage{amssymb}
\usepackage{amsthm}
\usepackage{graphicx}
\usepackage{hyperref}
\usepackage{geometry}
\usepackage{enumitem}
\usepackage{algorithm}
\usepackage{algorithmic}

\geometry{margin=1in}

\title{MoE Expert Models: A Comprehensive Analysis of Three Specialized Fairness-Aware Neural Networks}
\author{Fairness Machine Learning Project}
\date{\today}

\begin{document}

\maketitle

\section{Problem Formulation}

Let $\mathcal{C} = \{c_1, c_2, \dots, c_N\}$ denote a set of platforms (clients). Each platform $c_i$ holds a private local dataset $D_i$ that cannot be accessed by other platforms:
\begin{equation}
    D_i = \{(x_{i,j}, s_{i,j}, y_{i,j})\}_{j=1}^{|D_i|},
\end{equation}
where $x_{i,j} \in \mathbb{R}^d$ is the feature vector of applicant $j$ on platform $i$ (including personal attributes and possibly representations of unstructured data), $s_{i,j} \in \{0,1\}$ is a binary sensitive attribute (e.g., $0$ = female, $1$ = male), and $y_{i,j} \in \{0,1\}$ is the target label indicating whether the applicant receives a positive decision.

We study fairness with respect to $s$ and focus on both result fairness and procedural fairness. Let $f_\theta$ be a predictive model with parameters $\theta$. For an input $x$, the model outputs class probabilities $p = f_\theta(x) \in [0,1]^2$ and a predicted label $\hat{y} = \arg\max_c p_c$. Let $h_\theta(x)$ be the hidden representation and let $A_\theta(x) \in \mathbb{R}^d$ denote a normalized attribution ("attention") vector computed via a post-hoc interpreter (e.g., SHAP/IG/GradientSHAP), normalized so that $A_\theta(x) \ge 0$ and $\sum_{k=1}^d A_\theta(x)_k = 1$.

\paragraph{Result Fairness Metrics.} We use Demographic Parity (DP) and Equal Opportunity (EO) gaps:
\begin{align}
    \text{DP} &= \left| \mathbb{P}(\hat{y}=1\mid s=1) - \mathbb{P}(\hat{y}=1\mid s=0) \right|, \\
    \text{EO} &= \left| \mathbb{P}(\hat{y}=1\mid s=1, y=1) - \mathbb{P}(\hat{y}=1\mid s=0, y=1) \right|.
\end{align}

\paragraph{Procedural Fairness Metrics.} We consider three complementary notions derived from explanations and attributions. Let $\text{EQ}(x) \in \mathbb{R}_{\ge 0}$ denote a scalar explanation quality for instance $x$ produced by a fixed explainer. Define a top-$K$ rule: let $\tau_K$ be the $K$-th percentile of $\{\text{EQ}(x_j)\}_{j=1}^n$ over the evaluation set and
\begin{equation}
    \hat{q}_j = \mathbb{1}[\text{EQ}(x_j) \ge \tau_K].
\end{equation}
Let $G_s^K = \{ j : s_j = s,\; \hat{q}_j = 1 \}$ be the indices of top-$K$ high-quality explanations within group $s$.
\begin{itemize}
    \item \textbf{$\Delta$REF} (Ratio-based Explanation Fairness): difference in high-quality rates between groups
    \begin{align}
        \Delta\text{REF} &= \left| \mathbb{P}(\hat{q}=1\mid s=0) - \mathbb{P}(\hat{q}=1\mid s=1) \right|.
    \end{align}
    Smaller is better; $0$ indicates equal access to high-quality explanations.
    \item \textbf{$\Delta$VEF} (Value-based Explanation Fairness): difference in average explanation quality among top-$K$ within each group
    \begin{align}
        \Delta\text{VEF} &= \left|\; \frac{1}{|G_0^K|} \sum_{j\in G_0^K} \text{EQ}(x_j)\;-
        \frac{1}{|G_1^K|} \sum_{j\in G_1^K} \text{EQ}(x_j) \;\right|.
    \end{align}
    Smaller is better; $0$ indicates equal quality among the top-$K$ across groups.
    \item \textbf{ATT\_JSD}: Jensen–Shannon divergence between the group-level attribution distributions
    \begin{align}
        \text{ATT\_JSD} &= \text{JSD}(\bar{A}_0, \bar{A}_1)
        = \tfrac{1}{2}\,\text{KL}(\bar{A}_0 \Vert M) + \tfrac{1}{2}\,\text{KL}(\bar{A}_1 \Vert M), \\
        M &= \tfrac{1}{2}(\bar{A}_0 + \bar{A}_1).
    \end{align}
\end{itemize}

\paragraph{Problem Definition.} Given platforms $\mathcal{C}$ with local datasets $\{D_i\}_{i=1}^N$, learn a global model $f_\theta$ that maximizes predictive utility while mitigating result fairness gaps (DP/EO) and procedural fairness disparities (REF/VEF/ATT\_JSD), under data privacy constraints (no raw data sharing across platforms). A generic objective is
\begin{equation}
\label{eq:objective}
    \min_{\theta}\; \mathbb{E}_{(x,y)\sim D}\big[\mathcal{L}_{\text{CE}}(f_\theta(x), y)\big]
    + \lambda_{dp}\,\text{DP} + \lambda_{eo}\,\text{EO}
    + \lambda_{ref}\,\Delta\text{REF} + \lambda_{vef}\,\Delta\text{VEF} + \lambda_{att}\,\text{ATT\_JSD},
\end{equation}
where $D = \bigcup_{i=1}^N D_i$ conceptually denotes the union distribution (implemented in practice via local training/aggregation), and $\lambda_{\bullet} \ge 0$ control the utility–fairness trade-off. Lower values of DP/EO/REF/VEF/ATT\_JSD indicate better fairness.

\section{Introduction}

This document provides a detailed mathematical and architectural analysis of the three expert models implemented in the Mixture of Experts (MoE) framework for fairness-aware machine learning. The system consists of three specialized neural network experts, each designed to address different aspects of fairness in machine learning: utility optimization, result-driven fairness, and procedural fairness.

\section{Architecture Overview}

All three experts inherit from a common \texttt{ExpertBase} class and share a fundamental Multi-Layer Perceptron (MLP) backbone. The base architecture consists of:

\begin{itemize}
    \item Input layer: $d_{input}$ dimensions
    \item Hidden layer: $d_{hidden} = 8$ dimensions (default)
    \item Output layer: 2 dimensions (binary classification)
    \item Activation: LeakyReLU
    \item Dropout: 0.3 (applied to hidden layer)
\end{itemize}

The forward pass of each expert follows the pattern:
\begin{align}
    h &= \text{MLP}_{hidden}(x) \in \mathbb{R}^{d_{hidden}} \\
    \ell &= \text{MLP}_{output}(h) \in \mathbb{R}^{2} \\
    p &= \text{softmax}(\ell) \in \mathbb{R}^{2}
\end{align}

where $h$ represents the hidden representation, $\ell$ the logits, and $p$ the output probabilities.

\section{Expert 1: Utility-Focused Expert}

\subsection{Objective}
Expert 1 is designed to maximize predictive accuracy without explicit fairness constraints. It serves as a baseline utility-focused model.

\subsection{Architecture}
Expert 1 inherits directly from \texttt{ExpertBase} without additional components. The loss function is the standard cross-entropy loss:

\begin{equation}
    \mathcal{L}_{E1} = \text{CrossEntropy}(p, y) = -\sum_{i=1}^{n} \sum_{c=0}^{1} y_{i,c} \log(p_{i,c})
\end{equation}

where $y_{i,c}$ is the one-hot encoded true label for sample $i$ and class $c$, and $p_{i,c}$ is the predicted probability.

\subsection{Mathematical Formulation}
The complete loss computation is:
\begin{align}
    \mathcal{L}_{E1} &= \mathcal{L}_{CE} \\
    \mathcal{L}_{CE} &= -\frac{1}{n} \sum_{i=1}^{n} \log(p_{i,y_i})
\end{align}

where $y_i$ is the true class label for sample $i$.

\section{Expert 2: Result-Driven Fairness Expert}

\subsection{Objective}
Expert 2 addresses result-driven fairness by incorporating both representation alignment and demographic parity/equal opportunity constraints.

\subsection{Architecture}
Expert 2 extends \texttt{ExpertBase} with two additional hyperparameters:
\begin{itemize}
    \item $\lambda_{rep}$: Weight for representation alignment loss
    \item $\lambda_{fair}$: Weight for fairness loss
\end{itemize}

\subsection{Mathematical Formulation}

The total loss function combines three components:

\begin{equation}
    \mathcal{L}_{E2} = \mathcal{L}_{CE} + \lambda_{rep} \cdot \mathcal{L}_{rep} + \lambda_{fair} \cdot \mathcal{L}_{fair}
\end{equation}

\subsubsection{Representation Alignment Loss}
The representation alignment loss minimizes the distance between group mean embeddings:

\begin{align}
    \mu_0 &= \frac{1}{n_0} \sum_{i: s_i = 0} h_i \\
    \mu_1 &= \frac{1}{n_1} \sum_{i: s_i = 1} h_i \\
    \mathcal{L}_{rep} &= ||\mu_0 - \mu_1||_2^2
\end{align}

where $s_i \in \{0,1\}$ is the sensitive attribute for sample $i$, and $n_0, n_1$ are the counts of samples in each group.

\subsubsection{Fairness Loss}
The fairness loss combines demographic parity (DP) and equal opportunity (EO) gaps:

\begin{equation}
    \mathcal{L}_{fair} = \frac{\mathcal{L}_{DP} + \mathcal{L}_{EO}}{2}
\end{equation}

\textbf{Demographic Parity Gap:}
\begin{align}
    p_0 &= \frac{1}{n_0} \sum_{i: s_i = 0} p_{i,1} \\
    p_1 &= \frac{1}{n_1} \sum_{i: s_i = 1} p_{i,1} \\
    \mathcal{L}_{DP} &= |p_0 - p_1|
\end{align}

\textbf{Equal Opportunity Gap:}
\begin{align}
    p_0^{pos} &= \frac{1}{n_0^{pos}} \sum_{i: s_i = 0, y_i = 1} p_{i,1} \\
    p_1^{pos} &= \frac{1}{n_1^{pos}} \sum_{i: s_i = 1, y_i = 1} p_{i,1} \\
    \mathcal{L}_{EO} &= |p_0^{pos} - p_1^{pos}|
\end{align}

where $p_{i,1}$ is the probability of positive class for sample $i$, and $n_0^{pos}, n_1^{pos}$ are the counts of positive samples in each group.

\section{Expert 3: Procedural Fairness Expert}

\subsection{Objective}
Expert 3 addresses procedural fairness through attention alignment and adversarial debiasing techniques.

\subsection{Architecture}
Expert 3 extends \texttt{ExpertBase} with:
\begin{itemize}
    \item $\lambda_{attention}$: Weight for attention fairness loss
    \item $\lambda_{adv}$: Weight for adversarial debiasing loss
    \item Gradient Reversal Layer (GRL) for adversarial training
    \item Two adversarial networks: one for hidden representations, one for logits
\end{itemize}

\subsection{Mathematical Formulation}

The total loss function is:

\begin{equation}
    \mathcal{L}_{E3} = \mathcal{L}_{CE} + \lambda_{attention} \cdot \mathcal{L}_{attention} + \lambda_{adv} \cdot \mathcal{L}_{adv}
\end{equation}

\subsubsection{Attention Fairness Loss}
The attention fairness loss uses Jensen-Shannon Divergence (JSD) to measure the difference in attention distributions between groups:

\begin{align}
    \mathcal{L}_{attention} &= \text{JSD}(A_0, A_1) \\
    \text{JSD}(A_0, A_1) &= \frac{1}{2} \text{KL}(A_0 || M) + \frac{1}{2} \text{KL}(A_1 || M)
\end{align}

where $M = \frac{1}{2}(A_0 + A_1)$ is the average distribution, and $A_0, A_1$ are the mean attention weights for each group.

The attention weights are extracted using interpretability methods (SHAP, Integrated Gradients, LIME, or Gradient SHAP) and normalized:

\begin{equation}
    A_{i,j} = \frac{|\text{attribution}_{i,j}|}{\sum_{k=1}^{d} |\text{attribution}_{i,k}| + \epsilon}
\end{equation}

\subsubsection{Adversarial Debiasing Loss}
The adversarial loss prevents the model from learning sensitive attribute information:

\begin{equation}
    \mathcal{L}_{adv} = \frac{\mathcal{L}_{adv}^{hidden} + \mathcal{L}_{adv}^{logits}}{2}
\end{equation}

\textbf{Hidden Representation Adversarial Loss:}
\begin{align}
    h_{adv} &= \text{GRL}(h) \\
    \ell_{adv}^{hidden} &= \text{MLP}_{adv}^{hidden}(h_{adv}) \\
    \mathcal{L}_{adv}^{hidden} &= \text{CrossEntropy}(\ell_{adv}^{hidden}, s)
\end{align}

\textbf{Logits Adversarial Loss:}
\begin{align}
    p_{adv} &= \text{GRL}(p) \\
    \ell_{adv}^{logits} &= \text{MLP}_{adv}^{logits}(p_{adv}) \\
    \mathcal{L}_{adv}^{logits} &= \text{CrossEntropy}(\ell_{adv}^{logits}, s)
\end{align}

\subsubsection{Gradient Reversal Layer}
The GRL implements adversarial training by reversing gradients during backpropagation:

\begin{align}
    \text{GRL}(x) &= x \quad \text{(forward pass)} \\
    \frac{\partial \text{GRL}(x)}{\partial x} &= -\lambda \quad \text{(backward pass)}
\end{align}

\section{Implementation Details}

\subsection{MLP Backbone}
The shared MLP backbone uses the following architecture:
\begin{itemize}
    \item Linear layer: $d_{input} \rightarrow d_{hidden}$
    \item LeakyReLU activation
    \item Dropout (0.3)
    \item Linear layer: $d_{hidden} \rightarrow 2$
\end{itemize}

\subsection{Adversarial Networks}
Expert 3 includes two adversarial networks:

\textbf{Hidden Adversarial Network:}
\begin{itemize}
    \item Input: $d_{hidden}$ dimensions
    \item Hidden: $\max(4, d_{hidden}/2)$ dimensions
    \item Output: 2 dimensions (sensitive attribute prediction)
\end{itemize}

\textbf{Logits Adversarial Network:}
\begin{itemize}
    \item Input: 2 dimensions (logits)
    \item Hidden: 8 dimensions
    \item Output: 2 dimensions (sensitive attribute prediction)
\end{itemize}

\section{Training and Optimization}

Each expert is trained independently with its specialized loss function. The training process involves:

\begin{enumerate}
    \item Forward pass through the MLP backbone
    \item Computation of specialized loss components
    \item Backpropagation with appropriate gradient modifications (e.g., GRL for Expert 3)
    \item Parameter updates using standard optimization algorithms
\end{enumerate}

\section{Gating Network Architecture}

\subsection{Overview}
The Gating Network is responsible for selecting which expert to use for each input sample. It uses a REINFORCE-based policy gradient approach to learn optimal expert selection.

\subsection{Architecture}
The gating network consists of:
\begin{itemize}
    \item Input layer: $d_{input} + 3 \cdot d_{classes} + 3 + 1$ dimensions
    \item Hidden layer: 16 dimensions (default)
    \item Output layer: 3 dimensions (one for each expert)
    \item Activation: ReLU
    \item Temperature parameter: $\tau = 1.0$ for softmax scaling
\end{itemize}

\subsection{State Representation}
The gating network receives an augmented state vector:
\begin{equation}
    s = [x, p_1, p_2, p_3, \text{conf}_1, \text{conf}_2, \text{conf}_3, \text{disagree}]
\end{equation}

where:
\begin{align}
    \text{conf}_i &= \max(p_i) \quad \text{(confidence of expert } i\text{)} \\
    \text{disagree} &= \frac{1}{3} \left( \|p_1 - p_2\|_1 + \|p_1 - p_3\|_1 + \|p_2 - p_3\|_1 \right)
\end{align}

\subsection{Policy Formulation}
The gating network outputs a probability distribution over experts:
\begin{align}
    \ell &= \text{MLP}(s) \in \mathbb{R}^3 \\
    \pi(a|s) &= \text{softmax}(\ell/\tau) \in \mathbb{R}^3
\end{align}

\subsection{Action Sampling}
Expert selection follows a categorical distribution:
\begin{align}
    a &\sim \text{Categorical}(\pi(a|s)) \\
    \log \pi(a|s) &= \ell_a - \log \sum_{j=1}^3 \exp(\ell_j/\tau)
\end{align}

\subsection{Mixture Output}
The final prediction is a weighted combination of expert outputs:
\begin{equation}
    p_{final} = \sum_{i=1}^3 \pi_i \cdot p_i
\end{equation}

where $\pi_i$ is the gating probability for expert $i$ and $p_i$ is the output probability from expert $i$.

\subsection{Expert Routing}
During training, each sample is routed to its selected expert:
\begin{equation}
    p_{selected} = \begin{cases}
        p_1 & \text{if } a = 1 \\
        p_2 & \text{if } a = 2 \\
        p_3 & \text{if } a = 3
    \end{cases}
\end{equation}

During evaluation, the mixture output is used for consistent performance measurement.

\section{Training Algorithm}

\subsection{Two-Phase Training}
The MoE system uses a two-phase training approach:

\subsubsection{Phase 1: Expert Pretraining}
Each expert is pretrained independently for $\frac{T}{2}$ epochs:
\begin{align}
    \mathcal{L}_{total} &= \mathcal{L}_{E1} + \mathcal{L}_{E2} + \mathcal{L}_{E3} \\
    \mathcal{L}_{E1} &= \text{CrossEntropy}(p_1, y) \\
    \mathcal{L}_{E2} &= \text{CrossEntropy}(p_2, y) + \lambda_{rep} \mathcal{L}_{rep} + \lambda_{fair} \mathcal{L}_{fair} \\
    \mathcal{L}_{E3} &= \text{CrossEntropy}(p_3, y) + \lambda_{attention} \mathcal{L}_{attention} + \lambda_{adv} \mathcal{L}_{adv}
\end{align}

\subsubsection{Phase 2: Gate Training}
The gating network is trained using REINFORCE for $\frac{T}{2}$ epochs.

\subsection{REINFORCE Algorithm}
The gate training uses policy gradient with the following components:

\subsubsection{Reward Formulation}
The reward function balances utility improvement and fairness improvement:
\begin{equation}
    R = (U_2 - U_1) - (F_2 - F_1)
\end{equation}

where:
\begin{align}
    U &= \frac{\text{Accuracy} + \text{F1} + \text{AUC}}{3} \\
    F &= \frac{F_{result} + F_{procedure}}{2} \\
    F_{result} &= \frac{\text{DP} + \text{EO}}{2} \\
    F_{procedure} &= \frac{\text{REF} + \text{VEF} + \text{ATT}}{3}
\end{align}

\subsubsection{Baseline and Advantage}
A moving average baseline is used to reduce variance:
\begin{align}
    b_{t+1} &= \alpha \cdot b_t + (1-\alpha) \cdot R_t \\
    A_t &= R_t - b_t
\end{align}

\subsubsection{Policy Loss}
The policy loss includes entropy regularization and load balancing:
\begin{equation}
    \mathcal{L}_{gate} = -\mathbb{E}[\log \pi(a|s) \cdot A] - \lambda_{entropy} \mathcal{H}(\pi) + \lambda_{lb} \mathcal{L}_{lb}
\end{equation}

\subsubsection{Entropy Regularization}
The entropy term encourages exploration:
\begin{equation}
    \mathcal{H}(\pi) = -\sum_{i=1}^3 \pi_i \log(\pi_i + \epsilon)
\end{equation}

where $\epsilon$ is a small constant to avoid numerical issues.

\subsection{Load Balancing}
To ensure balanced expert usage, a KL divergence penalty is applied:
\begin{align}
    q_i &= \frac{\text{usage\_ma}_i + \epsilon}{\sum_{j=1}^3 (\text{usage\_ma}_j + \epsilon)} \\
    \mathcal{L}_{lb} &= \text{KL}(q \| \text{Uniform}(1/3))
\end{align}

where $\text{usage\_ma}_i$ is updated using exponential moving average:
\begin{equation}
    \text{usage\_ma}_i^{(t+1)} = \beta \cdot \text{usage\_ma}_i^{(t)} + (1-\beta) \cdot \pi_i^{(t)}
\end{equation}

\subsection{Expert Fine-tuning}
During gate training, experts are periodically fine-tuned to maintain their performance:
\begin{equation}
    \mathcal{L}_{fine-tune} = \mathcal{L}_{E1} + \mathcal{L}_{E2} + \mathcal{L}_{E3}
\end{equation}

This fine-tuning occurs every 100 epochs to prevent expert degradation during gate training.

\section{Evaluation Metrics}

\subsection{Utility Metrics}
\begin{align}
    \text{Accuracy} &= \frac{1}{n} \sum_{i=1}^n \mathbb{1}[\hat{y}_i = y_i] \\
    \text{F1} &= \frac{2 \cdot \text{Precision} \cdot \text{Recall}}{\text{Precision} + \text{Recall}} \\
    \text{AUC} &= \int_0^1 \text{TPR}(t) \cdot \text{FPR}'(t) \, dt
\end{align}

\subsection{Result Fairness Metrics}
\begin{align}
    \text{DP Gap} &= \left| \mathbb{E}[\hat{y}|s=0] - \mathbb{E}[\hat{y}|s=1] \right| \\
    \text{EO Gap} &= \left| \mathbb{E}[\hat{y}|s=0, y=1] - \mathbb{E}[\hat{y}|s=1, y=1] \right|
\end{align}

\subsection{Procedural Fairness Metrics}
\begin{align}
    \text{REF} &= \text{Representation Fairness} \\
    \text{VEF} &= \text{Value Fairness} \\
    \text{ATT} &= \text{Attention JSD between groups}
\end{align}

\subsection{Final Score}
The overall performance is evaluated using:
\begin{equation}
    \text{Final Score} = U - F_{result} - F_{procedure}
\end{equation}

\section{Implementation Details}

\subsection{Caching System}
The system supports expert and gate caching:
\begin{itemize}
    \item Expert weights are cached after pretraining
    \item Gate weights are cached after training
    \item Cache directory: \texttt{weights/moe\_experts/}
\end{itemize}

\subsection{Optimization}
\begin{itemize}
    \item Expert optimizer: Adam with learning rate $lr$ and weight decay
    \item Gate optimizer: Adam with learning rate $gate\_lr$ and no weight decay
    \item Momentum for baseline: $\alpha = 0.9$
    \item Momentum for usage tracking: $\beta = 0.9$
\end{itemize}

\subsection{Training Hyperparameters}
\begin{itemize}
    \item Total epochs: $T$ (default: 200)
    \item Expert pretraining: $T/2$ epochs
    \item Gate training: $T/2$ epochs
    \item Learning rates: $lr = 10^{-3}$, $gate\_lr = 10^{-3}$
    \item Regularization: $\lambda_{entropy} = 10^{-3}$, $\lambda_{lb} = 10^{-3}$
\end{itemize}

\section{Conclusion}

The three expert models provide complementary approaches to fairness in machine learning:

\begin{itemize}
    \item \textbf{Expert 1}: Pure utility optimization without fairness constraints
    \item \textbf{Expert 2}: Result-driven fairness through representation alignment and statistical parity
    \item \textbf{Expert 3}: Procedural fairness through attention alignment and adversarial debiasing
\end{itemize}

The gating network uses REINFORCE to learn optimal expert selection, balancing utility and fairness improvements. This multi-expert framework allows for flexible combination of different fairness approaches, enabling the system to adapt to various fairness requirements and trade-offs between accuracy and fairness.

\end{document}
